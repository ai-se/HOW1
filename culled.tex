In should be noted  that the standard 10-way cross-validation
procedure used  in data mining does not necessarily demonstrate G-causality.  In a N-way cross-val experiments, the data is divided into $N$ bins.  
$N-1$ bins are used for learning and the results are tested
on the  remaining bin.
This is repeated $N$ times, each time using a different bin for the test set.
This approach can not be used to demonstrate causality since, if the initial data is collected
over (say) $N$ months, then it is possible that the test set contains observations collected
{\em before}   items seen in the training set. 

Jiang et al.~\cite{me11f} and Lumpe et al.~\cite{me11f} propose a modification to cross-validation.
which lets a researcher
demonstrate G-causal defect effects in software engineering.
In this modification, we  divide project data  into $N$ bins, each of with contain data collected at similar times.
They then sort those bins (on the collection date) and use bins $i,j$ to build models that
are applied to bin  $k$ where 
$i<j<k$. Note that this procedure ensures that we are building theories on data seen
before the test data.

We adapt the work of Jiang,Lumpe et al.  as follows.