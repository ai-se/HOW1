\documentclass[conference]{IEEEtran}
\usepackage{colortbl}
\usepackage{booktabs}
\usepackage{subcaption}
\usepackage{algorithm}
\usepackage{algorithmicx}
\usepackage{algpseudocode}
\usepackage{tabulary}
\usepackage{bigstrut}
\setlength\fboxsep{1pt}
\setlength\fboxrule{1pt}
\usepackage{multicol}
\bstctlcite{IEEEexample:BSTcontrol}
\usepackage[table]{xcolor}
\usepackage{picture}
\newcommand{\keyword}[1]{\textit{#1}}
\newcommand{\quart}[4]{\begin{picture}(100,3)
  {\color{black}\put(#3,3){\circle*{4}}\put(#1,3){\line(1,0){#2}}}\end{picture}}
\usepackage{amsmath}
\usepackage{balance}
\usepackage{flushend}
\usepackage[english]{babel}
\usepackage{blindtext}
\usepackage{times}
\usepackage{cite}
\usepackage{hyperref}
\hypersetup{
  colorlinks = false,
  hidelinks = true
}
\setlength{\parindent}{0em}
\setlength{\parskip}{1em}

\begin{document}
  \title{Contrast Set Learning using WHAT}
  
  \author{\IEEEauthorblockN{Rahul Krishna}
    \IEEEauthorblockA{
      North Carolina State University, USA\\
      rkrish11@ncsu.edu}
    \and
    \IEEEauthorblockN{Tim Menzies}
    \IEEEauthorblockA{North Carolina State University, USA\\
      tim.menzies@gmail.com}}
  
  % make the title area
  \maketitle
  
  
  \begin{abstract}
 
  \end{abstract}
  \begin{IEEEkeywords}
    defect prediction, CART
  \end{IEEEkeywords}
  
\section{Introduction}
\section{Background}
\subsection{The Planning Algorithm}
The planning scheme makes use of WHERE to recursively split the data into smaller clusters. Following this, a nearest neighbour scheme is applied to identify pairs of nearby clusters. A projective plane is constructed with these clusters at the vertices to characterize the cluster pairs.

The mutation policy works by projecting the test instance onto the projective planes and identifying the the plane on which the test instance has the largest scalar projection. WHAT reflects over the vertices of the chosen hyperplane,  identifying the \textit{better} vertex among the two. A new instance is generated by mutating the attributes of the test instance towards the better vertex and away from the worse vertex. This process is repeated for all the test instances that are considered defective.
\section{Experimental Design}
The following section describes the experiments used to measure the performance of WHAT on 10 defect data sets and 6 performance prediction data sets.
\subsection{Data sets}
The defect data was obtained from the PROMISE repository. We investigated 32 releases from 11 open source Java projects defined by the metrics highlighted in: \textit{Ant} (1.5-1.7) 
\section{Experimental Results}
\section{Discussion}
\section{Threats to validity}
\section{Conclusion}
\section*{Acknowledgements}
%\begin{thebibliography}{10}
%\bibitem{test}
%\end{thebibliography}
\end{document}
