 \documentclass[conference]{IEEEtran}
\usepackage{colortbl}
\usepackage{booktabs}
\usepackage{subfig}
\usepackage{algorithm}
\usepackage{algorithmicx}
\usepackage{algpseudocode}
\setlength\fboxsep{1pt}
\setlength\fboxrule{1pt}
\usepackage{multicol}
\bstctlcite{IEEEexample:BSTcontrol}
\usepackage[table]{xcolor}
\definecolor{darkgray}{gray}{0.6}
\definecolor{Gray}{gray}{0.9}
\usepackage{picture}
\newcommand{\keyword}[1]{\textit{#1}}
\newcommand{\quart}[4]{\begin{picture}(50,3)
  {\color{black}\put(#3,3){\circle*{4}}\put(#1,3){\line(1,0){#2}}}\end{picture}}
\usepackage{amsmath}
\usepackage{balance}
\usepackage{flushend}
\usepackage{times}
\usepackage{cite}
\usepackage{hyperref}
\hypersetup{
  colorlinks = false,
  hidelinks = true
}
\setlength{\parindent}{0em}
\setlength{\parskip}{1em}

\begin{document}
  \title{An Instance Based Reasoning for Empirical Transfer Learning}
  
  \author{\IEEEauthorblockN{Rahul Krishna}
    \IEEEauthorblockA{
      North Carolina State University, USA\\
      rkrish11@ncsu.edu}
    \and
    \IEEEauthorblockN{Tim Menzies}
    \IEEEauthorblockA{North Carolina State University, USA\\
      tim.menzies@gmail.com}}
  
  % make the title area
  \maketitle
  
  
  \begin{abstract}
 
  \end{abstract}
  \begin{IEEEkeywords}
    defect prediction, CART
  \end{IEEEkeywords}
  
\section{Introduction}
\section{Motivating Example}
\section{Background}
\section{Instance Based Learning}

Assessing the quality of solutions to a real-world problem can be extremely hard. In several applications, researchers have models that can emulate the problem. Using these models it is possible to examine several scenarios in a short period of time, and this can be done in a reproducible manner. However, models aren't always the solution, as we shall see. 

There exist several problems where models are hard to obtain, or the input and output are related by complex connections that simply cannot be modeled in a reliable manner, or generation of reliable models take prohibitively long. Software defect prediction is an excellent example of such a case. Models that incorporate all the intricate issues that may lead defects in a product is extremely hard to come by. Moreover, it has been shown that models for different regions within the same data can have very different properties \cite{}. 

In this paper we propose the use of an instance based approach in place of the conventional model based approach. Instead of generating solutions and going back to a model to see if the solution has any value, we 

\subsection{Spectral Learning using WHERE}
The algorithm uses WHERE to recursively cluster the input data to identify subsets in the training data that a test instance can learn from. WHERE is a spectral learner which uses the FastMap heuristic to estimate the first principal component. It then recursively partitions the data into two halves along the median point of the projection on the first principal component, terminating when a half has less than $\sqrt{N}$ items.   

\subsection{The Planning Algorithm}
The planning scheme makes use of the clusters generated using the WHERE algorithm discussed above. Following this, a nearest neighbour scheme is applied to identify pairs of nearby clusters. A projective plane is constructed with these clusters at the vertices to characterize the cluster pairs.

The mutation policy works by projecting the test instance onto the projective planes and identifying the the plane on which the test instance has the largest scalar projection. The planner then reflects over the vertices of the chosen hyperplane, identifying the \textit{better} vertex among the two. Now, a new instance is generated by mutating the attributes of the test instance towards the better vertex and away from the worse vertex. This process is repeated for all the test instances that are considered defective.

\begin{figure}[!b]
\small
\begin{tabular}{|p{.95\linewidth}|}\hline
WHAT begins by \textit{clustering} the training data into similar subsets using WHERE which works as follows:
\begin{enumerate}
\item Find two distant points in that population; call them the {\em east} and {\em west} poles. 
\item Draw an axis of length $c$ between the poles. 
\item Let each point be at distance $a,b$ to the {\em east,west} poles.  Using the cosine rule, project each point onto the  axis  at $x=(a^2 + c^2 - b^2)/(2c)$.  
\item Using the median $x$ value, divide the population.
\item For each half that is larger than $\sqrt{N}$ of the original population, go to step 1.
\end{enumerate}

Note that the above requires a distance measure between sets of decisions: GALE uses the standard case-based reasoning measure defined by Aha et al.~\cite{aha91}. Note also that GALE implements step1 via  the FASTMAP~\cite{Faloutsos1995} linear-time
heuristic:
\begin{itemize}
\item Pick any point at random; 
\item Let {\em east} be the point furthest from that point; 
\item Let {\em west} be the point furthest from {\em east}.
\end{itemize}

These final sub-divisions found by this process are the {\em neighborhoods} that GALE will {\em perturb} as follows:
\begin{itemize}
\item Find the objective scores of the {\em east,west} poles in each neighborhood.
\item Using the continuous domination predicate of MOEA, find  the {\em better} pole. 
\item Perturb all points in that neighborhood by pushing them towards the better pole, by a distance  $c/2$ (recall that  $c$ is the distance between the poles).
\item Let generation $i+1$ be the combination of all pushed points from all neighborhoods.
\end{itemize}

From a formal perspective, GALE is an active learner~\cite{Dasgupta2005} that builds a piecewise linear approximation to the Pareto frontier~\cite{Zuluaga:13}.  For each piece, it then pushes the neighborhood up the local gradient.  This  approximation is built in the reduced dimensional space found the FASTMAP  Nystr\"om approximation to the first component of PCA~\cite{platt05}.
\\\hline
\end{tabular}
\caption{Inside GALE}\label{fig:gale}
\end{figure}

\subsection{Defect Prediction}


\begin{figure*}[htbp!]
  \renewcommand{\baselinestretch}{0.8}\begin{center}
    {\scriptsize
      \begin{tabular}{c|l|p{4.7in}}
        amc & average method complexity & e.g. number of JAVA byte codes\\\hline
        avg\_cc & average McCabe & average McCabe's cyclomatic complexity seen
        in class\\\hline
        ca & afferent couplings & how many other classes use the specific
        class. \\\hline
        cam & cohesion amongst classes & summation of number of different
        types of method parameters in every method divided by a multiplication
        of number of different method parameter types in whole class and
        number of methods. \\\hline
        cbm &coupling between methods &  total number of new/redefined methods
        to which all the inherited methods are coupled\\\hline
        cbo & coupling between objects & increased when the methods of one
        class access services of another.\\\hline
        ce & efferent couplings & how many other classes is used by the
        specific class. \\\hline
        dam & data access & ratio of the number of private (protected)
        attributes to the total number of attributes\\\hline
        dit & depth of inheritance tree &\\\hline
        ic & inheritance coupling &  number of parent classes to which a given
        class is coupled (includes counts of methods and variables inherited)
        \\\hline
        lcom & lack of cohesion in methods &number of pairs of methods that do
        not share a reference to an instance variable.\\\hline
        locm3 & another lack of cohesion measure & if $m,a$ are  the number of
        $methods,attributes$
        in a class number and $\mu(a)$  is the number of methods accessing an
        attribute, 
        then
        $lcom3=((\frac{1}{a} \sum_j^a \mu(a_j)) - m)/ (1-m)$.
        \\\hline
        loc & lines of code &\\\hline
        max\_cc & maximum McCabe & maximum McCabe's cyclomatic complexity seen
        in class\\\hline
        mfa & functional abstraction & number of methods inherited by a class
        plus number of methods accessible by member methods of the
        class\\\hline
        moa &  aggregation &  count of the number of data declarations (class
        fields) whose types are user defined classes\\\hline
        noc &  number of children &\\\hline
        npm & number of public methods & \\\hline
        rfc & response for a class &number of  methods invoked in response to
        a message to the object.\\\hline
        wmc & weighted methods per class &\\\hline
        \rowcolor{lightgray}
        defect & defect & Boolean: where defects found in post-release bug-tracking systems.
      \end{tabular}
    }
  \end{center}
  \caption{OO measures used in our defect data sets.  Last line is
    the dependent attribute (whether a defect is reported to  a
    post-release bug-tracking system).}\label{fig:ck}
\end{figure*}

\section{Experimental Design}
The following section describes the experiments used to measure the performance of WHAT on 10 defect data sets and 6 performance prediction data sets.
\subsection{Data sets}
The data was obtained from the PROMISE repository. For the defect data we investigated 32 releases from 11 open source Java projects defined by the metrics highlighted in \ref{}: \keyword{Apache Ant} (1.5 -- 1.7), \keyword{Apache Camel} (1.2 -- 1.6), \keyword{Apache Ivy} (1.1 -- 2.0), \keyword{JEdit} (4.1 -- 4.3), \keyword{Apache Log4j} (1.0 -- 1.2), \keyword{Apache Lucene} (2.0 -- 2.2), \keyword{PBeans} (1.0 and 2.0), \keyword{Apache POI} (2.0 -- 3.0), \keyword{Apache Synapse} (1.0 -- 1.2), \keyword{Apache Velocity} (1.4 -- 1.6), and \keyword{Apache Xalan-Java} (2.5 -- 2.7). Given the empirical nature of the data, it is important to design an experiment such that the planning phase uses only the \keyword{past} data to learn trends which can then be applied to the \keyword{future} data. Thus for our experiment we use data sets that have at least two consecutive releases. 
\begin{itemize}
\item To generate recommendations for a release $i$, the planner uses releases releases $(i-1)$ and $(i-2)$.
\item The predictor also uses releases $(i-1)$ and $(i-2)$. However, we use SMOTE with re-sampling in order to handle the class imbalance in the data and to prevent the predictor from using the same training data as the planner.
\end{itemize}


The performance prediction data set was obtained from \cite{}. The data set contains six examples of real world configurable systems: \keyword{Apache}, \keyword{LLVM}, \keyword{x264}, \keyword{Berkeley DB (written in C and Java}, and \keyword{SQLite}. The systems have different characteristics, different implementation languages, and different configuration mechanisms. The data set is a collection of all possible configurations (with SQLITE being a exception with only 4653 configurations). We performed a 5-fold cross validation study on this data set.

\subsection{Performance Assessment}

In order to assess the performance of the planner for the defect data set, we use the Cliffs Delta score to measure the probability that the number of bugs in test data before applying the planner is larger than after doing so. In other words, we use the delta score to measure the ability of the planner to effective reduce the number of defects. Given the untreated test instance labeled \textit{Before} of length \textit{M} and the treated instances labeled \textit{After}, the delta score is obtained as follows:
\begin{equation}
\delta = \frac{\#(Before>After) - \#(Before<After)}{M^2}
\end{equation}

In the context of our application, the $\delta$ attains a value of 1 if the number of defects after applying the treatments has reduced to zero, and 0 if there is no change.

For the performance prediction data set on the other hand, Cliff's delta measure is not directly applicable. We therefore assess the performance by measuring the gain, given by $\frac{Before}{After}$. The gain takes a value larger than 1 if the run times have been reduced, a value of 1 if there is no change, and a value less than 1 if run times have increased after applying the recommended changes. 

In addition to the above, we rank the different variants of the planning scheme to identify the best approach. We make use of the Scott-Knott procedure, recommended by \cite{} to compute the ranks. It works as follows: A list of treatments \textit{l} is sorted by the median score. The list \textit{l} is then split into sub-lists m, n in order to maximize the expected value of the differences in the observed performance before and after division. A statistical hypothesis test \textit{H} is applied on the splits \textit{m, n} to check if they are statistically different. If so, Skott-Knott then recurses on each division. In our paper, we use the A12 test and a non-parametric bootstrap sampling for hypothesis testing. 
\section{Experimental Results}
\begin{figure}[!t]
  \centering
\begin{minipage}{0.5\textwidth}
\renewcommand{\baselinestretch}{}
{\bf \scriptsize Ant:}


{\scriptsize \begin{tabular}{l@{~~~}l@{~~~}r@{~~~}r@{~~~}c}
      \arrayrulecolor{darkgray}
      \rowcolor[gray]{.9} \textbf{Rank} & \textbf{Treatment} & \textbf{Med} & \textbf{IQR} & \\
    1 &     Baseline &    0.03  &  0.0 & \quart{0}{0}{0}{62} \\
    \hline  2 & $\alpha=0.25$, Prune=50\% &    0.15  &  0.09 & \quart{4}{6}{7}{62} \\
    \hline  3 & $\alpha=0.5$, Prune=50\% &    0.25  &  0.1 & \quart{12}{6}{14}{62} \\
    \hline  4 &     $\alpha=0.25$ &    0.34  &  0.05 & \quart{18}{4}{20}{62} \\
    4 &   $\alpha=0.25$, weight &    0.36  &  0.08 & \quart{18}{5}{21}{62} \\
    4 & $\alpha=0.75$, Prune=50\% &    0.41  &  0.13 & \quart{21}{8}{24}{62} \\
    \hline  5 &    $\alpha=0.5$, weight &    0.46  &  0.18 & \quart{24}{12}{27}{62} \\
    5 &      $\alpha=0.5$ &    0.53  &  0.15 & \quart{27}{10}{32}{62} \\
    \hline  6 &     $\alpha=0.75$ &    0.64  &  0.17 & \quart{34}{11}{39}{62} \\
    6 &   $\alpha=0.75$, weight &    0.73  &  0.17 & \quart{38}{11}{45}{62} \\
    \hline \end{tabular}\\}


{\bf \scriptsize Camel}


{\scriptsize \begin{tabular}{l@{~~~}l@{~~~}r@{~~~}r@{~~~}c}
    \arrayrulecolor{darkgray}
    \rowcolor[gray]{.9} \textbf{Rank} & \textbf{Treatment} & \textbf{Med} & \textbf{IQR} & \\
1 &   Baseline &    0.16  &  0.03 & \quart{0}{3}{1}{111} \\
1 &   $\alpha=0.25$ &    0.17  &  0.06 & \quart{0}{7}{2}{111} \\
1 & $\alpha=0.25$, weight &    0.17  &  0.06 & \quart{0}{7}{2}{111} \\
\hline  2 & $\alpha=0.25$, Prune=50\% &    0.22  &  0.06 & \quart{5}{8}{9}{111} \\
\hline  3 &    $\alpha=0.5$ &    0.25  &  0.1 & \quart{3}{14}{13}{111} \\
3 &  $\alpha=0.5$, weight &    0.26  &  0.14 & \quart{7}{19}{14}{111} \\
\hline  4 & $\alpha=0.5$, Prune=50\% &    0.32  &  0.16 & \quart{10}{21}{22}{111} \\
\hline  5 & $\alpha=0.75$, weight &    0.38  &  0.19 & \quart{14}{25}{30}{111} \\
5 &   $\alpha=0.75$ &    0.4  &  0.12 & \quart{23}{16}{32}{111} \\
\hline  6 & $\alpha=0.75$, Prune=50\% &    0.47  &  0.15 & \quart{30}{19}{42}{111} \\
\hline \end{tabular}}\\


{\bf \scriptsize Ivy}


    {\scriptsize\begin{tabular}{l@{~~~}l@{~~~}r@{~~~}r@{~~~}c}
        \arrayrulecolor{darkgray}
        \rowcolor{Gray} \textbf{Rank} & \textbf{Treatment} & \textbf{Med} & \textbf{IQR} & \\
        1 &     Baseline &    0.06  &  0.0 & \quart{0}{0}{0}{79} \\
        \hline  2 & $\alpha=0.25$, Prune=50\% &    0.14  &  0.04 & \quart{4}{3}{6}{79} \\
        \hline  3 & $\alpha=0.5$, Prune=50\% &    0.2  &  0.06 & \quart{9}{5}{11}{79} \\
        \hline  4 & $\alpha=0.75$, Prune=50\% &    0.27  &  0.14 & \quart{10}{12}{17}{79} \\
        \hline  5 &     $\alpha=0.25$ &    0.41  &  0.06 & \quart{27}{5}{29}{79} \\
        5 &   $\alpha=0.25$, weight &    0.42  &  0.06 & \quart{27}{6}{30}{79} \\
        \hline  6 &    $\alpha=0.5$, weight &    0.55  &  0.11 & \quart{35}{9}{41}{79} \\
        6 &      $\alpha=0.5$ &    0.57  &  0.09 & \quart{37}{7}{43}{79} \\
        \hline  7 &   $\alpha=0.75$, weight &    0.61  &  0.08 & \quart{42}{7}{46}{79} \\
        7 &     $\alpha=0.75$ &    0.62  &  0.04 & \quart{46}{3}{47}{79} \\
        \hline \end{tabular}}\\


{\bf \scriptsize Jedit}


  {\scriptsize\begin{tabular}{l@{~~~}l@{~~~}r@{~~~}r@{~~~}c}
      \arrayrulecolor{darkgray}
      \rowcolor{Gray} \textbf{Rank} & \textbf{Treatment} & \textbf{Med} & \textbf{IQR} & \\
      1 &   Baseline &    0.05  &  0.0 & \quart{0}{0}{0}{75} \\
      \hline  2 & $\alpha=0.25$, Prune=50\% &    0.11  &  0.05 & \quart{2}{4}{4}{75} \\
      \hline  3 & $\alpha=0.5$, Prune=50\% &    0.16  &  0.1 & \quart{3}{8}{8}{75} \\
      \hline  4 &   $\alpha=0.25$ &    0.2  &  0.08 & \quart{10}{6}{11}{75} \\
      4 & $\alpha=0.25$, weight &    0.21  &  0.1 & \quart{7}{8}{12}{75} \\
      \hline  5 & $\alpha=0.75$, Prune=50\% &    0.32  &  0.15 & \quart{15}{11}{21}{75} \\
      5 &  $\alpha=0.5$, weight &    0.33  &  0.05 & \quart{19}{4}{22}{75} \\
      5 &    $\alpha=0.5$ &    0.34  &  0.18 & \quart{18}{14}{23}{75} \\
      \hline  6 &   $\alpha=0.75$ &    0.56  &  0.2 & \quart{27}{16}{40}{75} \\
      6 & $\alpha=0.75$, weight &    0.56  &  0.26 & \quart{29}{20}{40}{75} \\
      \hline \end{tabular}}\\
\end{minipage}
\end{figure}
\begin{figure}
\begin{minipage}{0.5\textwidth}
  \renewcommand{\baselinestretch}{}
{\bf \scriptsize Log4j}


  {\scriptsize\begin{tabular}{l@{~~~}l@{~~~}r@{~~~}r@{~~~}c}
      \arrayrulecolor{darkgray}
      \rowcolor{Gray} \textbf{Rank} & \textbf{Treatment} & \textbf{Med} & \textbf{IQR} & \\
      1 & $\alpha=0.5$, Prune=50\% &    0.1  &  0.06 & \quart{1}{11}{5}{172} \\
      1 & $\alpha=0.25$, Prune=50\% &    0.11  &  0.04 & \quart{1}{8}{7}{172} \\
      1 & $\alpha=0.25$, weight &    0.1  &  0.03 & \quart{3}{6}{5}{172} \\
      1 &  $\alpha=0.5$, weight &    0.1  &  0.13 & \quart{0}{24}{5}{172} \\
      1 &   Baseline &    0.11  &  0.04 & \quart{0}{7}{7}{172} \\
      1 &   $\alpha=0.25$ &    0.12  &  0.06 & \quart{1}{11}{9}{172} \\
      \hline  2 &    $\alpha=0.5$ &    0.15  &  0.09 & \quart{3}{17}{14}{172} \\
      2 & $\alpha=0.75$, Prune=50\% &    0.15  &  0.05 & \quart{9}{9}{14}{172} \\
      2 & $\alpha=0.75$, weight &    0.17  &  0.22 & \quart{9}{40}{18}{172} \\
      2 &   $\alpha=0.75$ &    0.22  &  0.16 & \quart{14}{30}{27}{172} \\
      \hline \end{tabular}}\\


{\bf \scriptsize Lucene}


{\scriptsize\begin{tabular}{l@{~~~}l@{~~~}r@{~~~}r@{~~~}c}
    \arrayrulecolor{darkgray}
    \rowcolor{Gray} \textbf{Rank} & \textbf{Treatment} & \textbf{Med} & \textbf{IQR} & \\
    1 & $\alpha=0.25$, weight &    0.03  &  0.03 & \quart{0}{4}{1}{148} \\
    1 &  $\alpha=0.25$ &    0.05  &  0.04 & \quart{1}{6}{4}{148} \\
    \hline  2 &   $\alpha=0.5$ &    0.09  &  0.09 & \quart{6}{13}{10}{148} \\
    2 & $\alpha=0.25$, Prune=50\% &    0.09  &  0.03 & \quart{7}{5}{10}{148} \\
    2 &  Baseline &    0.1  &  0.01 & \quart{10}{2}{12}{148} \\
    2 & $\alpha=0.5$, weight &    0.14  &  0.06 & \quart{10}{9}{18}{148} \\
    \hline  3 & $\alpha=0.5$, Prune=50\% &    0.19  &  0.08 & \quart{21}{12}{25}{148} \\
    \hline  4 & $\alpha=0.75$, weight &    0.23  &  0.15 & \quart{22}{23}{31}{148} \\
    4 &  $\alpha=0.75$ &    0.26  &  0.1 & \quart{27}{15}{36}{148} \\
    4 & $\alpha=0.75$, Prune=50\% &    0.28  &  0.13 & \quart{30}{19}{39}{148} \\\hline \end{tabular}}\\


{\bf \scriptsize Poi}


{\scriptsize\begin{tabular}{l@{~~~}l@{~~~}r@{~~~}r@{~~~}c}
    \arrayrulecolor{darkgray}
    \rowcolor{Gray} \textbf{Rank} & \textbf{Treatment} & \textbf{Med} & \textbf{IQR} & \\
    1 & $\alpha=0.25$, Prune=50\% &    0.09  &  0.15 & \quart{0}{9}{3}{62} \\
    1 &     Baseline &    0.09  &  0.05 & \quart{2}{3}{3}{62} \\
    1 & $\alpha=0.5$, Prune=50\% &    0.12  &  0.13 & \quart{1}{8}{5}{62} \\
    \hline  2 &     $\alpha=0.25$ &    0.24  &  0.3 & \quart{5}{19}{13}{62} \\
    2 &   $\alpha=0.25$, weight &    0.25  &  0.32 & \quart{4}{20}{14}{62} \\
    \hline  3 & $\alpha=0.75$, Prune=50\% &    0.37  &  0.45 & \quart{4}{29}{21}{62} \\
    3 &    $\alpha=0.5$, weight &    0.49  &  0.5 & \quart{9}{32}{29}{62} \\
    \hline  4 &     $\alpha=0.75$ &    0.53  &  0.47 & \quart{16}{30}{32}{62} \\
    4 &      $\alpha=0.5$ &    0.53  &  0.28 & \quart{18}{18}{32}{62} \\
    4 &   $\alpha=0.75$, weight &    0.61  &  0.43 & \quart{22}{27}{37}{62} \\
    \hline \end{tabular}}  \\


{\bf \scriptsize Velocity}


  {\scriptsize\begin{tabular}{l@{~~~}l@{~~~}r@{~~~}r@{~~~}c}
      \arrayrulecolor{darkgray}
      \rowcolor{Gray} \textbf{Rank} & \textbf{Treatment} & \textbf{Med} & \textbf{IQR} & \\
      1 & $\alpha=0.25$ &    0.08  &  0.03 & \quart{0}{4}{1}{140} \\
      1 & $\alpha=0.25$, weight &    0.09  &  0.05 & \quart{0}{7}{3}{140} \\
      \hline  2 & Baseline &    0.13  &  0.04 & \quart{6}{6}{9}{140} \\
      \hline  3 & $\alpha=0.5$ &    0.16  &  0.09 & \quart{9}{13}{13}{140} \\
      3 & $\alpha=0.25$, Prune=50\% &    0.17  &  0.04 & \quart{13}{6}{15}{140} \\
      3 & $\alpha=0.5$, weight &    0.19  &  0.09 & \quart{12}{13}{18}{140} \\
      \hline  4 & $\alpha=0.5$, Prune=50\% &    0.22  &  0.11 & \quart{15}{16}{22}{140} \\
      4 & $\alpha=0.75$, weight &    0.25  &  0.1 & \quart{19}{15}{27}{140} \\
      4 & $\alpha=0.75$ &    0.28  &  0.2 & \quart{19}{30}{31}{140} \\
      4 & $\alpha=0.75$, Prune=50\% &    0.28  &  0.09 & \quart{24}{13}{31}{140} \\
      \hline \end{tabular}}\\




{\bf \scriptsize Xalan}


  {\scriptsize\begin{tabular}{l@{~~~}l@{~~~}r@{~~~}r@{~~~}c}
      \rowcolor{Gray} \textbf{Rank} & \textbf{Treatment} & \textbf{Med} & \textbf{IQR} & \\
      1 & $\alpha=0.75$, weight &    0.31  &  0.14 & \quart{0}{31}{13}{170} \\
      1 & $\alpha=0.25$, weight &    0.32  &  0.1 & \quart{2}{22}{15}{170} \\
      1 &   $\alpha=0.25$ &    0.32  &  0.06 & \quart{13}{14}{15}{170} \\
      \hline  2 &    $\alpha=0.5$ &    0.34  &  0.11 & \quart{6}{25}{20}{170} \\
      2 &   Baseline &    0.36  &  0.13 & \quart{4}{30}{24}{170} \\
      2 &  $\alpha=0.5$, weight &    0.35  &  0.12 & \quart{6}{28}{22}{170} \\
      2 & $\alpha=0.5$, Prune=50\% &    0.36  &  0.12 & \quart{11}{27}{24}{170} \\
      2 & $\alpha=0.25$, Prune=50\% &    0.38  &  0.1 & \quart{13}{23}{29}{170} \\
      2 & $\alpha=0.75$, Prune=50\% &    0.37  &  0.21 & \quart{2}{47}{27}{170} \\
      2 &   $\alpha=0.75$ &    0.38  &  0.17 & \quart{6}{39}{29}{170} \\
      \hline \end{tabular}}
 \end{minipage}  
   \end{figure}


% \section{Discussion}
% \section{Threats to validity}
% \section{Conclusion}
% \section*{Acknowledgements}
%\begin{thebibliography}{10}
%\bibitem{test}
%\end{thebibliography}
\end{document}
